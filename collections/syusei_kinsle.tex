\documentclass[a4paper,12pt]{article}
%% Sets page size and margins
\usepackage[a4paper,top=3cm,bottom=2cm,left=3cm,right=3cm,marginparwidth=1.75cm]{geometry}
\usepackage[dvipdfmx]{graphicx}
% \usepackage{graphicx}
\usepackage[utf8]{inputenc}
\usepackage{indentfirst}
\usepackage{amsmath}
\usepackage{amssymb}
\usepackage{adjustbox}
\usepackage{natbib}
\usepackage{authblk}
\usepackage{arydshln}
% \usepackage[caption = false]{subfig}
\usepackage{appendix}
% \usepackage{changepage} 
\usepackage{lscape}


\title{教員付加価値からみた教員の役割について−日本の小学生を例にして}
\author{伊藤寛武 田端紳}
\author{
伊藤寛武
\thanks{慶應義塾大学政策メディア研究科博士課程, Email: itouhrtk@keio.jp
%Graduate School of Media and Governance, Keio University, 5322 Endo,
%Fujisawa-shi, Kanagawa 252-0882 Japan, Email: itouhrtk@keio.jp.
}
\and 
田端紳 
\thanks{慶應義塾大学経済学研究科修士課程, Email: stabata@gs.econ.keio.ac.jp
%Graduate School of Economics, Keio University, Mita 2-15-45, Minato-ku, Tokyo  108-8345 Japan, Email: stabata@gs.econ.keio.ac.jp.
}
}
\date{March 31, 2019}
\begin{document}
\maketitle
\begin{abstract}
子どもの能力の成長に教員は大きな影響を与えうるとして、多くの研究者や政策担当者がそれぞれの国のデータを用いて学力に与える教員の影響について検証してきた。その一方で日本の教員を対象に行われた分析は非常に少なく、子どもたちの能力の成長に教員がどの程度影響を持っているのかはよくわかっていない。本稿では日本のとある都市における個票データを用いて教員付加価値を推定し、教員が子どもたちの能力の成長にどのような影響を持っているかを記述的に分析した。その際には、学力のみならず学習方略・非認知能力・学級の雰囲気を対象にして教員付加価値を求めた。その結果、教育成果の種類に応じて個別の教員の影響の大きさは大きく異なることが分かった。特に、学力や学習方略に対する個別の教員の影響は生徒の変動のうち1\%弱であった一方で、非認知能力や学級の雰囲気に対する個別の教員が子どもの教育成果に与える影響は生徒の変動のうち5\%から8\%ほどであった。さらに、教員付加価値間の相関係数から、学力を伸ばすことが得意な教員と非認知能力・学習方略を伸ばすことが得意な教員は異なりうるという結果を得た。以上の結果から、個別の教員は学力ではなく非認知能力に対して重要な役割を担っている可能性が示唆された。
% > 特に、学力や学習方略に対する個別の教員の影響は生徒の変動のうち1\%弱であった一方で、非認知能力や学級の雰囲気に対する個別の教員の影響は生徒の変動のうち5\%から8\%ほどであった。
%  大きい小さいではなく、具体的に
\end{abstract}

JEL Classification: I21\\
キーワード:教育成果, 教員効果

\newpage
% Korpershoek, H., Harms, T., de Boer, H., van Kuijk, M., & Doolaard, S. (2016). A meta-analysis of the effects of classroom management strategies and classroom management programs on students’ academic, behavioral, emotional, and motivational outcomes. Review of Educational Research, 86(3), 643-680.
\section{はじめに}
% todo:直す
% -> ここでそもそも教育経済学では、教員がインパクトがある!って言う研究がせいぜいであり、具体的に子どもにとって教員がどういう存在なのかはよく分かっていないことにも触れる
% -> 多くの教育研究者や政策担当者が、教員の質が子どもの教育成果にどのような影響を与えているのかという問いに関心を寄せてきた。そもそも教員が子どもに知識を伝達するという現代の教育システムの中では教員が子どもに影響を与える事自体は自明視されてきた一方、その影響がどの程度であるかと言う点については未だ解明されていないことも多い。また近年の研究では古くからの研究の妥当性への批判も多くなされ、より大規模なデータを用いて検証した研究が増えてきている。
多くの教育研究者や政策担当者が教員の質が子どもの教育成果にどのような影響を与えているのかという問いに関心を寄せ、現代においては教員付加価値に関する研究としてその議論は深化している。子どもは学校という場所を通して多くの時間を教員と共に過ごす。そのため、教員が子どもに対して強い影響力を持つ可能性を多くの研究が指摘してきたが\citep{hanushek1986economics,hanushek2006school, ammermuller2005schooling,hojo2012factors}、近年その議論は教員付加価値についての研究として発展を見せている\cite{kane, chetty, lefgan}。教員付加価値とは子どもの能力の成長に対する個々の教員の貢献を評価したものである。その教員付加価値を推定することは手法上の問題を抱えるものの\citep{jackson2014teacher}、教員付加価値に関する主要な研究は教員の子どもの学力に対する因果的な影響が大きいことを支持している。


% -> そして、政策的問題としても教員効果がどの様な性質を持っているか重要な問題である。子どもの教育成果に対する教員のパフォーマンス評価を行う際には教員の教育付加価値に着目する分析が多いが(\citep{jackson2014teacher})が、その教員付加価値がどのような性質を持っているかについては未だ研究途上である。確かに、実験的状況もしくは擬似実験的な状況を用いて教員付加価値が因果効果を含むという点までは先行研究が指摘している(\cite{chetty2014measuring})。その一方で、教員付加価値が教員のどの様な特徴に由来するものなのかなどはよくわかっていない。例えば学力について高いパフォーマンスをあげている教員は、非認知能力や学級運営においても高いパフォーマンスを上げているのだろうか。その間にトレードオフの関係があれば、学力向上を意図して良い教員を雇用することはむしろ非認知能力などに対しては負の効果を持つかもしれない。
% 教員の能力は多次元的なもの
% Alexander, Entwisle, and Thompson (1987), Ehrenberg, Goldhaber, and Brewer (1995), Downey and Shana (2004),Jennings & DiPrete (2010), and Mihaly et al. (2013) find evidence that teachers have effect on non-test-score measures of student skills. Also, Koedel (2008) estimates high-school teacher effects on graduation. 
% 教員付加価値に関する研究は子どもの学力を対象にして付加価値の推定を行うものが多いが、子どもにとって教員は学力以上の存在でありうる。学力だけが子どもにとって重要な人的資本ではない。非認知能力と呼ばれる学力テストでは測ることのできない能力も子どもの人的資本の重要な要素の一つとして頻繁に言及されている。重要性が指摘される、\cite{jackson2014teacher}。
% ここ書き直すtodo
しかし、教員が子どもに影響を与えているのは学力だけではなく非認知能力をも含んだ多次元的な教育成果である可能性があり、その点についてはいまだに分かっていないことも多い。学校の中で教員はただ授業をして子どもの学力を伸ばすだけの存在ではない。授業や学級の中で生徒とのコミュニケーションを通じて、非認知能力などの学力以外の能力の形成にも教員は影響を及ぼしている可能性がある。\cite{todo}などは、非認知能力に対する教員付加価値を推定しその因果効果の存在などを報告している。

% 「非認知能力に関する研究が指し示すのはそういう可能性」


% -> また、日本においては教育経済学や教育社会学の分野における教員の効果について統計的な分析をした研究は非常に少ない\footnote{国際学力調査を用いた教育生産関数の文脈で教員の影響を分析する場合は存在する\citep{hojo2012factors, hojo2012determinants}。しかし、これらは教員の効果を分析することを目的とした研究ではない。}。\cite{二木美苗2017子ども}が日本におけるある程度の規模のデータセットを用いた数少ない教員効果に関する研究であるが、あくまで都道府県レベルでの分析であり教員個人の教育効果を分析するものではない。教員の役割を含めた教育システムや学校を取り巻く状況は国によって大きな違いがあり、海外での研究成果をどの程度日本においてそのまま適用可能かどうかはよくわかっていない。
さらに、日本を対象にした教員の効果に対する定量的研究は非常に少なく、知見の蓄積が必要である。教員の役割を含めた教育システムや学校を取り巻く状況は国によって大きな違いがあり、海外での研究成果をどの程度日本においてそのまま適用可能かどうかはよくわかっていない。それにもも関わらず、研究が少ない。国際学力調査を用いた教育生産関数の文脈で教員の影響を分析する場合は存在する\citep{hojo2012factors, hojo2012determinants}。しかし、これらは教員の効果を分析することを目的とした研究ではない。。\cite{二木美苗2017子ども}が日本におけるある程度の規模のデータセットを用いた数少ない教員効果に関する研究であるが、あくまで都道府県レベルでの分析であり教員個人の教育効果を分析するものではない。



%->  本稿では日本のとある市における教員調査のデータを用いて、子どもの教育成果への教員の影響について分析を行う。調査は3年に渡って行われ継時的に追跡可能になる様に全ての教員には固有のIDが割り振られた。また調査の際には、教員への質問紙調査と同時に生徒への学力テストおよび質問紙調査が行われた。質問紙調査の中では子どもの非認知能力や学習方略加えて学級の様子について尋ねており、そのため本稿では子どもの教育成果について多面的に把握することが出来た。
% 本稿では教員効果の分析にあたって、教員付加価値の推定を行う。子どもの教育成果(の変化)に対して教員がどれほど貢献しているかを表す教員付加価値は、\cite{kane2008estimating}の研究に始まり幅広くなされてきた\footnote{米国における教員効果の検証および関連する政策に関するレビューとして\cite{jackson2014teacher}をあげることができる。}。その中で、教員付加価値が因果的な効果を持ちうることや\citep{kane2008estimating,chetty2014measuring}、その教員付加価値は学力のみならず非認知能力などを対象にしても同様に存在すること\citep{jackson2014teacher}などが知られている。一方で、比類される日本の教員についての研究は管見の限り未だ出ておらず、日本の中で教員が子どもの教育成果にどの程度貢献しているかについてほとんどわかっていない。例えば教員は子どもの人的資本蓄積に対してどの程度の役割を持っているのだろうか。もしくは、子どもの学力を伸ばせる「有能」な教員を雇用することは、どの様な効果を産みうるのだろうか。それらの問いに対する答えは現状ほぼ出ていない。そのため、本稿では\cite{kane2008estimating}の手法を用いて時間に対して固定的な教員付加価値を推定し、その教員付加価値の性質について記述的な分析を行った。
本稿では小学校の教員を対象にして教員付加価値の推定を行い、その記述的な分析を行うことで子どもの人的資本蓄積において教員が果たしている役割を分析することを試みる。教員付加価値の推定には日本のある市町村における全ての子どもと教職員を対象に3年に渡って行われた調査のデータを用いた。その調査では、毎年4月に子どもに対する学力調査と質問紙調査及び教員に対する質問紙調査を行っている。特に、子どもに対する質問紙調査の中では非認知能力や学習方略に加えて学級の様子について尋ねており、本稿では学力に限らない多面的な教育成果に対する教員付加価値を推定をすることができた。得られた教員付加価値の分布や相関などを見ていくことで、教員付加価値がどの様な性質を持っているかを考えた。



% -> その結果として、教育成果の種類に応じて個別の教員の影響の大きさは大きく異なることが分かった。特に、学力や学習方略に対する個別の教員の影響は生徒の変動のうち1\%弱であった一方で、非認知能力や学級の雰囲気に対する個別の教員の影響は生徒の変動のうち5\%から8\%ほどであった。さらに、教員付加価値間の相関係数から、学力を伸ばすことが得意な教員と非認知能力・学習方略を伸ばすことが得意な教員は異なりうるという結果を得た。一方で、年齢や性別といった教員の観察可能な特徴は推定された教員付加価値との間に統計的に有意な相関関係はなかった。
結果として、教員の影響力は教育成果の種類に応じて異なり、個々の教員は非認知能力や学級の雰囲気に対してより高い影響力を持つことが分かった。学力や学習方略に対する教員付加価値の標準偏差は生徒の変動のうち1\%弱であった一方で、非認知能力や学級の雰囲気に対する教員付加価値の標準偏差は生徒の変動のうち5\%から8\%ほどであった。そのため個々の教員の影響力は非認知能力や学級の雰囲気作りにおいてより強いことが示唆された。さらに、学力を伸ばすことが得意な教員と非認知能力を伸ばすことが得意な教員は異なることが分かった。教員付加価値間の相関を計算すると、学力の教員付加価値間の相関及び非認知能力・学級の雰囲気の教員付加価値間の相関は統計的に有意に正であるのに対して、学力と非認知能力の教員付加価値間の相関は小さく統計的に有意ではなかった。最後に、高い教員付加価値を持つ教員の具体的特徴を見つけることはできなかった。すなわち、年齢や性別といった教員の観察可能な特徴と推定された教員付加価値との間に統計的に有意な相関関係はなかった。


本論文の構成は以下の通りである。第 \ref{litelature} 節では先行研究を概観する。第 \ref{data}節ではデータについて述べる。第 \ref{results} 節では推定結果について述べる。第 \ref{conclusion}節では本稿における結論を述べる。

\section{先行研究\label{litelature}}

% -> 教員が子どもの教育成果に与える影響への関心は古くから存在し、多くの研究がなされてきた。教育経済学の分野で行われた初期の研究については、\cite{hanushek2006teacher}にまとめられている。\cite{hanushek2006teacher}では、それまでに行われてきた教員の質に関する研究を大きく3つの文脈に類型化し、それぞれの文脈で得られている知見を整理している。その1つ目は労働市場一般における教員の立ち位置(賃金や志望倍率)を用いるものである。例えば日本を対象にした研究である\cite{二木美苗2017子ども}は、労働市場における他職に比べた教員の人気度を用いて分析を行なっている。これらの研究では総じて、他の産業と比較して優秀な人材がいるということが子どもの成績を上げる可能性を検証している。
教員の子どもの教育成果への有効性についての研究は古くから存在し\footnote{教育経済学の分野で行われた初期の研究については、\cite{hanushek2006teacher}にまとめられている。}、。\cite{hanushek2006teacher}によれば3つの文脈に類型化できる。1つ目は労働市場一般における教員の立ち位置(賃金や志望倍率)を用いて教員の分析を行うものである。

2つ目は教員の属性が子どもの教育成果に与える影響を分析するものである。ここで言う教員の性質というのは具体的には教員免許の有無や経験年数などを指す。代表的なものとして\cite{Jackson2009aejae,clotfelter2010teacher}などを挙げる事ができる。
% jackson(2009)

3つ目は子どもの教育成果に対する教員付加価値を計算して分析するものである。教員付加価値とは子どもの教育成果の成長に対する教員の貢献を指した言葉であり\footnote{\cite{jackson2014teacher}は近年の教員の付加価値に関する研究動向をまとめている。}、もとより観察不可能な概念である。そのため、多くの場合は教員ごとの教員付加価値を推定することになるが、この教員付加価値をどの様に定式化するかについては必ずしも決まりきった方法があるわけではない。近年は\cite{kane2008estimating}に示された経験ベイズ的な手法を使って教員付加価値を推定する研究が多く\cite{chetty, lefgun}、これらの研究は総じて教員付加価値が子どもの教育成果に対して一定の重要性を担っていることを報告している。ただし、教員付加価値の推定には多くの仮定が含まれ、その仮定の妥当性を巡った議論がなされていることには注意が必要である\citep{rothstein, kinsler}。
% 教員付加価値の研究は進展を見せている。
%  * 推定方法
%  * 推定方法の前提の検証
% %  * その因果性の確認

しかし、教員付加価値の研究は学力を中心に進展しており、学力以外の教育成果に対する教員付加価値に対する研究には蓄積が少ない。上述の研究を始め多くの教員付加価値を扱った研究はテストのスコアに対する付加価値を分析しているが、子供の人的資本蓄積において非認知能力などの学力で表現できない能力の重要性が指摘されて久しい。しかし、非認知能力の向上への個々の教員の有効性について検証した研究は少ない。数少ない例として、\cite{jackson}は非認知能力に対する教員付加価値を推定しておりその有効性を示している。しかしその結果がどれほど一般的であるか、特に日本の教育システムでの同様であるかなどは分かっておらず、更なる研究が必要とされている。
% あとで後ろに「学力を中心に考えられる教員付加価値だが、少なくともこのデータを対象にした時には非認知の方が大きい」みたいに書く?

加えて、教員付加価値などで定式化される教員の能力は多次元的な概念であるのにも関わらず、その能力間の関係についても議論の蓄積が成されていない。非認知能力に対する教員付加価値を計算する研究が出てきているように、教員付加価値は学力のみならず様々な教育成果に対して考えることが出来る。認知能力を伸ばすことに長けた教員が、非認知能力を伸ばすことに長けている保証はない。若くは同じ認知能力であっても国語を伸ばすことに長けた教員が算数を伸ばすことにも長けているとは限らない。この様な教員の能力の多義性に着目し分析をしている研究は少ない。\cite{lefgun}では複数の科目間に基底して存在する認知能力全体に対する教員付加価値を計算している。\cite{goldhaber2013good}では米国ノースカロライナ州のデータをもちいて、小学校レベルでの教員の付加価値の科目間相関を検証した。その論文ではリーディング科目と数学科目の付加価値相関係数が0.7程度であることを報告している。また上述の\cite{jackson}では、認知能力と非認知能力の教員付加価値の間の相関が弱いことを報告しているが、この研究は中学3年生と比較的高年齢の子どもを対象にしている。


最後に、日本を対象にした教員の付加価値についての研究は非常に少ないことについて詳述する必要がある。国際学力調査を用いた教育生産関数の文脈で教員の影響を分析する場合は存在する\citep{hojo2012factors, hojo2012determinants}。しかし、これらは教員の効果を分析することを目的とした研究ではない。。\cite{二木美苗2017子ども}が日本におけるある程度の規模のデータセットを用いた数少ない教員効果に関する研究であるが、あくまで都道府県レベルでの分析であり教員個人の教育効果を分析するものではない。


% 以上のこれまでの教員研究の流れにおいて、本稿は3つ目の教員の付加価値に関して分析を行なうものである。特に本稿では、教員付加価値を計算する対象の教育成果として、学力のみならず非認知能力や学習方略・学級の雰囲気を対象に分析を行うことにある。
% 本稿における貢献は、教員の子どもに与える影響として学力のみならず、非認知能力や学級における雰囲気(classroom climate)を考えたことにある。教員が子どもに与える影響として最も直感的且つ最も調査されてきたのは学力に対する影響である一方、教員が影響を及ぼすのは学力だけではないだろう。教員が子どもひとりひとりに指導していくことを考えると、教員は例えば子どものパーソナリティや非認知能力などにも影響を与えうる。また、ほとんどの国では生徒をクラスに分けてクラスごとに授業などを行なっていくシステムを採用しているが、そのクラスルームにおけるコミュニケーションの雰囲気などにも教員の影響があるのではないかと十分に考えることができる。


\section{データ\label{data}}
本稿では2016年−2018年(調査の実施は2015年から行われている)にかけて行われた日本のとある県Sにおける学力テスト及び質問紙調査のデータを用いる。テストは毎年4月に行われ、テストと同時に非認知能力や生徒の普段の生活を尋ねる質問紙調査を行った。テストを受けた生徒にはユニークなIDが振られ、過年度で同一生徒を特定することが可能な設計になっている。調査の対象になったのは、S県における政令指定都市を除く全ての市町村の公立小学校及び公立中学校に所属する小学4年生から中学3年生までの生徒である。およそ生徒数では30万人程度、学校数では1064の学校(小学校が708校、中学校が356校)の生徒が参加した。ただし後述するように本項では教員のデータを用いるために、その中でもT市のみのデータを取り扱った。調査は毎年4月に行われる。日本のおける教育制度は4月を年度のはじめとすることが多く、そのため上記において$t$年に行われた調査は$t-1$年の生徒の実態を反映していると考えることができる。
% T市の学校の数とかも書くし、本来はそれを中心にする


上記の調査に加え、S県の市町村の一つであるT市にて教員に質問紙調査を2016年から2018年にかけて行なった。これは教員の普段の授業での実践を尋ねるもので、教員一人一人にユニークなIDが振られ過年度で同一教員を特定することが可能な設計になっている。ただし日本における公立学校の人事制度においてある教員は必ずしも同じ学校に所属し続けるわけではなく、T市以外の市町村への転出も多い。また全ての教員が毎年学級を一つ担当するわけでもなく、中には副担任などの形で自分の担当学級をもたない教員もいる。これらの理由から、複数年度で担当クラスを持っている教員はかなり少なくなってしまうことに注意をしなければいけない。


また今回分析に用いたのは小学校での教員データだけで、中学校での教員データは分析に用いなかった。その理由としてある教員の中学校での担当クラスは小学校での担当学級と質的に異なることを挙げることができる。日本の教育システムにおいては、多くの小学校の先生は一つの学級を受け持ち全ての科目を教えることになる\footnote{ただし習熟度別学級の実施などにより、一部科目において複数の学級を担当するような状況は存在する。}。一方で、多くの中学校では先生は専門科目をもち、複数の学級で授業を行うことになる。そのため、小学校の先生は学級の全ての状況に対しての関係性が比較的強いと考えることができるが、中学校の先生はある学級を受け持つ教員は複数おり誰に責任があるのか不明瞭である。以上の理由から、本稿では小学校の先生でのみ分析を行なった。


\ref{table:des_stat}と\ref{table:des_stat_tch}にはデータで用いる変数についての記述統計(平均値、標準偏差、サンプルサイズ)を示した。\ref{table:des_stat}には子どもを単位とする変数の記述統計を示し、\ref{table:des_stat_tch}には教員を単位とする変数の記述統計を示した。以降ではデータで用いる変数について記述する。

\begin{landscape}
\begin{table}[htbp]
\centering
\begin{adjustbox}{max width=1.5\textwidth}
\input{table/des_stat.tex}
\end{adjustbox}
\caption{記述統計}
\label{table:des_stat}
\begin{flushleft}
\footnotesize{
注)本表では、本稿で用いている変数の記述統計を示している。各々のセルでは、対応する変数について「平均(サンプルサイズ、標準偏差)」という記述方法を用いている。データ単位は全て、子ども一人である。
}
\end{flushleft}
\end{table}
\end{landscape}


\begin{table}[htbp]
\centering
\begin{adjustbox}{max width=1.5\textwidth}
\input{table/des_stat_tch.tex}
\end{adjustbox}
\caption{記述統計教員}
\label{table:des_stat_tch}
\begin{flushleft}
\footnotesize{
注)本表では、本稿で用いている変数の記述統計を示している。各々のセルでは、対応する変数について「平均(サンプルサイズ、標準偏差)」という記述方法を用いている。データ単位は全て、教員一人である。
}
\end{flushleft}
\end{table}


\subsection*{認知能力}
認知能力を表す値として、国語と算数の学力の推定値を用いる。S県学力調査では小学生に対して国語と算数のテストを実施している。それらの結果から、IRT(Item Response Theory)を用いて個人の科目ごとの学力の推定値を算出した。加えて本項では分析にあたって全ての学力の推定値を年度・学年および科目ごとに平均0・分散1に標準化を行った。この操作によって、実質的に学年内での相対的な高低を標準化された学力は表すことになる。このIRTによって算出された学力推定値の標準化された値を以降では認知能力の値として分析の対象とする。

\subsection*{非認知能力及び学習方略}
S県学力調査ではテストとは別に質問紙による子どもの生活状況を尋ねる調査を行なっており、その調査の中で子どもの非認知能力や学習方略を調査している。本稿ではそのうち、(1) セルフコントロール \citep{duckworth2013really}, (2) 自己効力感 \citep{pintrich1991manual}, (3)学習方略\citep{佐藤純1998学習方略の使用と達成目標及び原因帰属との関係}を教育成果として分析の対象にする。これらの値はそれぞれ複数の質問項目(表\ref{table:item}・表\ref{table:item2})を足しあげることで得られる。さらに分析においては学年ごとに平均0・分散1に標準化を行ない、学年内での相対的な高低を表す値に直した。
% (3) 勤勉性 \citep{barbaranelli2003questionnaire}
% todo:全てに聞いているわけではない

\subsection*{学級の雰囲気}
本稿では子どもが所属する学級の雰囲気を教員付加価値を計算する対象の1つとして用いる。生徒質問紙調査においては子どもの前年の学級における教師及び友人との関係性について調査している(表\ref{table:item2})。本項ではこれらの項目を教員によって干渉可能な学級における雰囲気を表す質問項目と考える。学級の雰囲気を教育成果として考えることはあまり一般的ではないが、一方で生徒とのコミュニケーションを通じて教員は受け持つ学級の雰囲気にも影響を与えると考えられる。本稿で用いる学級の雰囲気を表す質問を表\ref{table:item2}に記した。質問は全て4件法によってなされ、それらの項目全てを足し合わせて学級の雰囲気を表す変数とした。さらに分析においては学年ごとに平均0分散1に標準化を行なった。
%  todo: 本当は先行研究をあげる

% 文章を直す.正当化をする。

\subsection*{その他の変数}
調査において子どもの家庭での状況などを子どもに尋ねて調査している。そのうち、本稿では(1)週あたりの通塾時間(8件法)及び(2)家庭にある本の冊数(5件法)を用いる。これらは共に子どもの家庭に社会経済的地位(Socio-economic Status、以降SESと表記)を表すプロキシとして用いる。前者は経済資本を表す変数であり、後者は文化資本を表す変数である。
% 文章を直す.論文をあげて正当化をする。

\begin{landscape}
\begin{table}[htbp]
\begin{adjustbox}{max width=1.5\textwidth}
\input{table/item.tex}
\end{adjustbox}
\caption{質問リスト:非認知能力}
\label{table:item}
\end{table}
\end{landscape}


\begin{landscape}
\begin{table}[htbp]
\begin{adjustbox}{max width=1.5\textwidth}
\input{table/item2.tex}
\end{adjustbox}
\caption{質問リスト:学習方略、学級の雰囲気}
\label{table:item2}
\end{table}
\end{landscape}

\section{推定戦略と推定結果\label{results}}
\subsection{教員付加価値の算出\label{methodology}}
ある生徒の教育成果は次の様な教育生産関数で表せられるとする。

 \begin{equation}
 \label{basic}
  y_{it} = \alpha y_{it-1} +  X_{it} \boldsymbol{\beta} +  \mu_j + \theta_{c} + \epsilon_{it}
 \end{equation}
$ y_{it}$は$t$年における生徒$i$ の教育成果を表す。本稿では教育成果として、学力(国語、算数)の他に非認知能力(自己効力感、セルフコントロール)や学習方略・学級の雰囲気を考える。$X_{it} $は観察可能な生徒$i$の特徴を表し、本稿ではSESの代理変数である家庭にある本の冊数および通塾時間を用いる。$\theta_{c} $は生徒$i$が所属する学級$c$の効果を表し、$\mu_j$ は学級$c$を担当する教員$j$の付加価値を表し、$\epsilon_{it}$は生徒$i$の$t$年におけるその他の要因全てを表す。$\mu_j$・  $\theta_{c}$・ $\epsilon_{it}$は全て観察不可能であり、互いに独立であることを仮定する。

% todo:学級は年度ごとに違うカウントすることを説明

% $ I_{i, j, t} $ はt年における生徒iが教員jのクラスだった場合に1の値を取りそれ以外の場合は0をとる割り当てを表す変数である。$ T_{j ,t} $は教員jのt年における効果を表し、今後推定された教員の効果と述べた時は式\ref{basic}で推定された$T_{j, t}$を用いることとする。

% -> 式\ref{basic} では$y_{it}$を説明するために前期の能力を表す$y_{it-1}$をコントロールしており、そのため式\ref{basic}は極めて簡易的ながら付加価値モデルと広く呼ばれるモデルに含まれる\citep{todd2003specification}。生徒の能力を規定する要因は実に多くの要因が考えられるが、そのうち生徒自身の生来的能力やこれまで蓄積された能力若しくは家庭のSESなど時間変動しないであろう要因をコントロールし、生徒の能力の成長を変動として取り出し分析することを付加価値モデルでは意図している。
式\ref{basic}は付加価値モデルと広く呼ばれるモデルの一つである\citep{todd2003specification}。前期の能力を表す$y_{it-1}$をコントロールすることで、これまで蓄積された能力若しくは家庭のSESなど時間変動しない要因の影響を取り除くことを意図している。そのため、用いる変動は生徒の能力の成長を用いることになる。教員付加価値はこの生徒の能力の成長に対する教員の貢献を評価することになる。
% 何をコントロールしているかをここで書く

% -> さてこの場合、教員付加価値$\mu_{j}$をどの様に推定するかが大きな課題となる。教員付加価値に関する研究で頻繁に用いられるのは、\cite{kane2008estimating}の手続きに従い、$\mu_{j}$を求める手法である。Kaneらの手法では式\ref{basic}を推定し、その残差を用いて$\mu_{j}$の推定値を計算する。すなわち、式\ref{basic}を$s_{it} =   y_{it} - \alpha y_{it-1} -  X_{it}\boldsymbol{\beta} = \mu_j + \theta_{c} + \epsilon_{it} $と書き直せば、$s_{it}$は$\mu_{j}$という求めたい値に$\theta_{c}$ と$\epsilon_{it}$というノイズが加わった変数とみなせる。この時、ある学級$c$はある教員$j$によって担任されるため、$s_{it}$を生徒iが所属するクラスで平均した値$s_{jc}$は
% \begin{eqnarray*}
% s_{jc}  &=& \frac{\sum_{i, i \in I_{c}} s _ {it}}{\|I_{c}\|}\\
% &=& \mu_{j} + \theta_{c} + \frac{\sum_{i, i \in I_{c}} \varepsilon _ {it}}{\|I_{c}\|} \\
% &=& \mu_{j} + v_{jc}
% \end{eqnarray*}
% となる。ただし学級$c$に所属する生徒の集合を$I_c$で表している。この$s_{jc}$は教員付加価値$\mu_j$についての信号とみなせるため、教員$j$の教員付加価値の期待値$\mu_j$はある重み$\alpha_{jc}$を用いて

% \begin{eqnarray*}
% E(\mu_j|s_{jc}) = \sum_{c, c \in I_{j}} \alpha_{jc}  s_{jc}\\
% \end{eqnarray*}
% で表される\footnote{重み$\alpha_{jc}$の計算については\cite{kane2008estimating}を参照せよ。}。
% todo: ある重み?→任意のウエイト?:αについての説明も少し足したほうがよい。
本稿では教員付加価値$\mu_{j}$を\cite{kane2008estimating}の手続きに従って求める。Kaneらの手法では式\ref{basic}を推定し、その残差を用いて$\mu_{j}$の推定値を計算する。すなわち、式\ref{basic}を$s_{it} =   y_{it} - \alpha y_{it-1} -  X_{it}\boldsymbol{\beta} = \mu_j + \theta_{c} + \epsilon_{it} $と書き直せば、$s_{it}$は$\mu_{j}$という求めたい値に$\theta_{c}$ と$\epsilon_{it}$というノイズが加わった変数とみなせる。この時、ある学級$c$はある教員$j$によって担任されるため、$s_{it}$を生徒iが所属するクラスで平均した値$s_{jc}$は
\begin{eqnarray*}
s_{jc}  &=& \frac{\sum_{i, i \in I_{c}} s _ {it}}{\|I_{c}\|}\\
&=& \mu_{j} + \theta_{c} + \frac{\sum_{i, i \in I_{c}} \varepsilon _ {it}}{\|I_{c}\|} \\
&=& \mu_{j} + v_{jc}
\end{eqnarray*}
となる。ただし学級$c$に所属する生徒の集合を$I_c$で表している。この$s_{jc}$は教員付加価値$\mu_j$についての信号とみなせるため、教員$j$の教員付加価値の期待値$\mu_j$はある重み$\alpha_{jc}$を用いて
\begin{eqnarray*}
E(\mu_j|s_{jc}) = \sum_{c, c \in I_{j}} \alpha_{jc}  s_{jc}\\
\end{eqnarray*}
で表される\footnote{重み$\alpha_{jc}$の計算については\cite{kane2008estimating}を参照せよ。}。



% -> 以上の手法には、複数の教育成果で共通する教員の付加価値を評価した\cite{lefgren2012using}や教員の付加価値の時間による変化を許容する\cite{chetty2014measuring}などの幾つかの派生系が存在する。また非常に似たような手法として、教員の固定効果を教員付加価値として用いる研究も存在する\footnote{Kaneらの手法が教員固定効果の計算と大きく異なる点として、$s_{it}$の信頼度で教員付加価値を評価しているという点を挙げることができる。例えば5人しかいないクラスを担当する教員の固定効果と30人いるクラスを担当する教員の固定効果ではその信頼度に大きく差があり、Kaneらの手法ではその点を考慮に入れて計算をしていると言うことができる} 。本稿では基本的に\cite{kane2008estimating}に従い教員付加価値を計算するが、算出された教員付加価値の頑健性を確認するため教員固定効果および\cite{chetty2014measuring}\footnote{上述した通り、複数年度T市に在籍する教員は少なく\cite{chetty2014measuring}の方法で計算した場合非常にサンプルが落ちてしまう。この点を考慮して補助的な扱いとした。}に従った計算も同時に行った。
教員付加価値を求める手法には幾つかの派生形があるが、手法に対する頑健性を示すために本稿でも複数の手法で教員付加価値を計算した。\cite{kane2008estimating}が示した教員付加価値を推定する方法は、ある教員が担当するクラスや生徒数は有限であることから必ずしも一致性などの統計的な良い性質が保証されているわけではない。また教員付加価値に期待される性質も状況によって異なりうる。\cite{kane2008estimating}の手法では教員付加価値は時間を通じて一定であることを前提にしているが、例えば新任の教員が業務経験を通じて徐々にその能力を高めていくことなどを考慮に入れようとすれば教員付加価値が時間を通じて変化しうるものとしてモデル化する必要がある。そのため、教員付加価値を推定する手法には複数の教育成果で共通する教員の付加価値を評価した\cite{lefgren2012using}や教員の付加価値の時間による変化を許容する\cite{chetty2014measuring}などの幾つかの派生系が存在する。さらに、非常に似たような手法として、教員の固定効果を教員付加価値として用いる研究も存在する\footnote{Kaneらの手法が教員固定効果の計算と大きく異なる点として、$s_{it}$の信頼度で教員付加価値を評価しているという点を挙げることができる。例えば5人しかいないクラスを担当する教員の固定効果と30人いるクラスを担当する教員の固定効果ではその信頼度に大きく差があり、Kaneらの手法ではその点を考慮に入れて計算をしていると言うことができる} 。本稿では基本的に\cite{kane2008estimating}に従い教員付加価値を計算するが、算出された教員付加価値の頑健性を確認するため教員固定効果および\cite{chetty2014measuring}\footnote{上述した通り、複数年度T市に在籍する教員は少なく\cite{chetty2014measuring}の方法で計算した場合非常にサンプルが落ちてしまう。この点を考慮して補助的な扱いとした。}に従った計算も同時に行った。


% ディフェンスしなくちゃいけないこと。教員効果の効果の与え方をkinslerに従ってもう少し議論する。下で書く。

  % 教員の効果と学級の効果を識別できない。この場合学級の効果とは二パターンあって、学級特有の字変動ショックと学級固有の特徴の2つを考える必要がある。前者はよくわかんない(kisnlerは犬が試験中吠えて集中力が切れたなどを考えている)。後者は、例えば、クラス一丸となって勉強を教え合ったとか、ものすごい問題児がいるとか。特にもし教員と生徒の構成が時間を通じて一定だった場合(持ち上がり)にはこの問題は大きい。すなわち、もしそのようであるとき、学級固有の特徴は教員のクラス割り当てと強く関係があるからである。そこで今回この持ち上がりがどれぐらいあるかを検討したところ、非常に少なかった。そのため上記の問題はある年の教員とクラスの特養の識別ができない、という程度に考えて良い

式\ref{basic} による教員効果の推定の重要な限界として、教員のクラス割り当てについての外生性の問題を挙げることができる。上記の手法で教員付加価値をバイアスなく得るためには、モデルのクラス効果$\theta_c$及び誤差項を含むその他の全ての要因$\epsilon_{it}$と教員の割り当ての間に内生性がないことを仮定する。そこでは生徒の観察不可能な特徴と教員割り当ての間に相関がないことも仮定しているが、この仮定の妥当性については十分に注意する必要がある。例えば、経験の浅い教員の授業負担を軽くするためにSESの高い生徒を優先的に割り当てているとする。そのような生徒が塾などの外部の教育リソースを活用して能力を伸ばしているといった場合、担当教員の教育効果を高く見積もってしまう可能性がある。以降の分析ではそのような内生性がないことを仮定するが、その点について十分に注意を払い結果を解釈していく必要がある。
% roththein testに言及して、「ヤンないけど重要な限界だよね〜」って書いておく?
% 学校の効果やクラスルーム効果との識別の問題がある。即ち、教員の配置が必ずしも外生的に起きていない可能性、また教員が学校に所属しているからそもそも識別ができないかも。後者については、後々の分析で学校固定効果や個人固定効果をコントロールする中で対処を試みるが、前者については重要な問題である(roththein testについて言及。ただし今回は教員に関するサンプルが少なく棄却できたとしても偽陽性の問題が強いため、論文の限界として強く認識をするに留める)。
 
% 以下2つは教員固定効果を用いていた場合の問題 
% また別の限界として教員と学校の効果の識別の問題を挙げることができる。教員は基本的に学校に所属して指導を行う。そのため、ある教員の教育効果はその教員個人だけによって齎されたわけではなく、学校全体の取り組みの結果として齎されたものである可能性がある。例えば、ある学校は教育意識の高い地域に所在しその結果高学年になると学校のみならず通塾などをすることで学力を伸ばすといった場合、上記で推定した教員効果はこの所在値の特徴による効果も含んでしまう。
% さらに教員と学級効果の識別の問題がある。上記で言及した教員とクラス割り当てにも関連するが、上記式で推定した教員効果ではその年の学級の効果を含みうる。例えば、あるクラスにおいて教員の能力関係なく自然発生的にクラス内での勉強の教えあいなどが発生して学力上がるといったケースがあったとする。上記で推定されたある年の教員効果はこのピアエフェクトをも含む。この問題に対して\cite{chetty2014measuring}とか\cite{rockoff2004impact}などの多くの教員の効果に関する研究は実験的環境で推定された教員効果が確かに教員の因果効果であったことを報告している。調査対象の市町村はもちろん日本においてその様な研究は管見の限りは無いため上記の議論をそのまま当てはめることには限界はあるものの、ある程度の教員効果としての妥当性があるものとして扱う。ただし、教員の因果効果として解釈することは慎重に検討しなければいけない。
 
さらに学級効果$\theta_c$の独立性の仮定は問題を含みうる。特にこの問題が重要になってくるのは、年度が変わってもある教員($j'$と表記する)は同じクラスを持ち上がって担当し、教員に割り当てられた生徒の構成が時間を通じて一定のケースの場合である(この様な教員割り当てを「持ち上がり」とここでは表記する)。「持ち上がり」が教員$j'$に発生している時、教員効果$\mu_{j'}$とクラス効果$\theta_c$は識別することができない。ある教員$j'$の付加価値の推定値は、ある同じクラスの観察不可能な特徴による効果を含むことになり、上記の問題はより深刻化する。日本の教育システムにおいて、ある教員とあるクラスの割り当てがどの程度持続するかは自治体や学校に依存する。クラス替えが毎年発生し教員にクラスを毎年割り当てし直す学校もあれば、2年から3年程度同じクラスを教員に割り当て続ける学校もある。T市においては持ち上がりクラスは全部で6クラスしかなく全体の1\%と非常に稀にしか起きていなかった。そのため、クラスの持ち上がりの問題はほとんど発生していないと考えることができる。
%  todo:結果を表Xに記した


    
\subsection{教員付加価値の分布\label{estimation}}
% -> 式\ref{basic}に基づき教員付加価値$\mu_j$を上述の手法を用いて推定を行った。個別の教員ごとにその値は求められるため推定された教員付加価値全てを具体的に示すことは難しいが、代わりに推定された教員付加価値のヒストグラムおよび密度関数(カーネル密度推定)を図\ref{table:variance}に示した。学力や非認知能力・学級の雰囲気といった教育成果の種類に依らず概ね教員付加価値は正規分布に近い形で分布していることがわかる。
式\ref{basic}に基づき推定された教員付加価値$\mu_j$の分布を図\ref{fig:decompose}に示し、異なる手法で求められた教員付加価値間の相関を表\ref{table:robust}に示した。図\ref{fig:decompose}においては教員付加価値のヒストグラムを棒グラフで示し、カーネル密度推定によって求められた密度関数を線グラフによって示した。さらに、表\ref{table:robust}では、\cite{kane2008estimating}で求めた教員付加価値とその他の手法で求めた教員付加価値の間にある相関を示している。各行は計算の対象となった教育成果を表す。TODO列においては前年度の教育成果をコントロールしない定式化の下で求めた教員付加価値との相関を示している。TODO列では教員固定効果\footnote{教員を表すダミー変数に係る係数の推定値を教員固定効果として求めた}との間にある相関を示した。TODO列では\cite{chetty2014measuring}の方法で求めた教員付加価値との間になる相関を示している。


% -> TODO:推定された教員付加価値は正規分布に近い形で分布を描いた。極端に能力が低い教員が孤立して存在するわけではなさそう。


% -> ただし、教員付加価値を求めるにあたっては幾つかの手法が存在する。もし手法によって求められる教員付加価値に大きな差があればどの手法を採用するかによって議論が大きく異なりうる。そのため、教員付加価値の推定手法の選択の妥当性は重要な問題である。
% ただし本稿においては、複数の方法や異なる定式化で計算された教員付加価値間の相関係数が高いことが確認されたため、教員付加価値は頑健に推定されたと考えられる。具体的には、推定値の手法や定式化に対する頑健性を確認するため、表\ref{table:robust}に図\ref{fig:decompose}で求めた教員付加価値と異なる定式化や推定方法で求めた教員付加価値の間にある相関を示した。教員付加価値間の相関係数はいずれも0.65から0.99の間で非常に高い値をとり、またいずれも1\%の水準で統計的に有意であった。すなわち、どのような手法や定式化を用いたとしても推定された教員付加価値は同様の傾向を示している。本稿では概ね\cite{kane2008estimating}に従った手法を用いて議論を行うが、その手法の選択は議論の結果には強い影響を及ぼすことはないと考えられる。
%  todo:手法の細かい種類についても書く
%  また教員付加価値は概ね頑健に推定された。
% -> また複数の方法や異なる定式化で計算された教員付加価値間の相関が高いことが確認されたため、教員付加価値は頑健に推定されたと考えることができる。
教員付加価値は推定の方法や定式化によらず頑健に推定された。すなわち、表\ref{table:robust}の全ての列全ての行において相関はも1\%の水準で統計的に有意に正であり、その係数は低くとも0.68((1)列の学習方略)よりも大きい。この結果は、どの様な推定方法や定式化をにおいても教員付加価値は同じ様傾向を示していることを表す。もし手法によって求められる教員付加価値に大きな差があればどの手法を採用するかによって議論が大きく異なりうるが、用いたデータの上ではその懸念は当たらない。そのため、手法の選択は本稿における議論の結果には強い影響を及ぼすことはないと考えられる。ただし、ここで確認した関係はあくまで相関であり推定された教員付加価値の水準値とは異なることには注意が必要である。



\begin{figure}
\centering
\includegraphics[width=0.45\textwidth]{./fig/distribution_zkokugo_level.pdf}
\includegraphics[width=0.45\textwidth]{./fig/distribution_zmath_level.pdf}
\includegraphics[width=0.45\textwidth]{./fig/distribution_zstrategy.pdf}
\includegraphics[width=0.45\textwidth]{./fig/distribution_zselfefficacy.pdf}
\includegraphics[width=0.45\textwidth]{./fig/distribution_zselfcontrol.pdf}
\includegraphics[width=0.45\textwidth]{./fig/distribution_zgood_class.pdf}
\label{fig:decompose}
\caption{推定された教員付加価値の分布}
\end{figure}



\begin{table}[htbp]
\begin{adjustbox}{max width=1.5\textwidth}
\input{table/corr_va_model.tex}
\end{adjustbox}
\caption{推定手法間の教員付加価値の相関}
\label{table:robust}
\begin{flushleft}
\footnotesize{
注)本表では、\cite{kane2008estimating}に従って求めた教員付加価値と、異なる手法や定式化で求めた教員付加価値の間の相関係数を計算している。各々の列は教員付加価値を求めるにあたって用いた手法や定式化を表し、各々の行は教員付加価値を求める対象とした教育成果を表す。第2列に表しているのは、\cite{kane2008estimating}(表内では「KS」と表記)に従いながらコントロール変数に前年度の教育成果の値を用いたなかった場合の相関係数を示している。第3列は固定効果を用いて推定した教員付加価値との相関係数を示している。第4列から第6列までは \cite{chetty2014measuring}(表内では「CFR」と表記)の手法を用いて推定した教員付加価値との相関係数を示している。「***」・「**」・「*」はそれぞれ1\%・5\%・ 10\%の水準で統計的に有意であることを表す。
}
\end{flushleft}
\end{table}


% -> それでは、子どもにとって教員付加価値はどれほど重要なのだろうか。広く信じられているように教員は子どもの教育成果に対して大きな影響を与えているのだろうか。その点について考えるために、表\ref{table:variance}には教員付加価値の標準偏差を示した。比較のために固定効果を用いて推定された教員付加価値の標準偏差および、子どもやクラス平均の標準偏差を同時に示した。また視覚的に確認するために図\ref{fig:kasane}では一部の教育成果を対象にして生徒の教育成果と教員付加価値のヒストグラムを同時に示した。
次に、教員の影響の大きさを評価するために、表\ref{table:variance}には教員付加価値の推定値の標準偏差を示した。(todo)列には\cite{kane2008estimating}の手法で求めた教員付加価値の標準偏差を教育成果ごとに記している。比較のために、他の手法で求めた教員付加価値や元々の記述統計も表\ref{table:variance}には同時に示している。(todo)列には固定効果法によって求めた教員付加価値の標準偏差を示した。(todo)列には教育成果の元々の標準偏差を示した。\ref{data}節で示している通り各教育成果に対しては標準化を行なっているため、ここでの値はほとんど1に近い\footnote{1と微量に異なる値になるのは、サンプルの脱落などが存在するからである}。(todo)列には教育成果のクラス平均の標準偏差を示した。


\begin{table}[htbp]
\begin{adjustbox}{max width=1\textwidth}
\input{table/des_stat_teacher_va.tex}
\end{adjustbox}
\caption{教員付加価値の標準偏差}
\label{table:variance}
\begin{flushleft}
\footnotesize{
注)本表では求めた教員付加価値の標準偏差を示している。各々の行は教員付加価値を求める対象とした教育成果を表す。2列目では本稿で主に用いている教員付加価値の標準偏差を示した。3列目では、比較のために固定効果推定で求めた教員付加価値の標準偏差の値を示した。4列目では生徒の教育成果の標準偏差示した。5列目ではクラス平均の標準偏差を示した。6列目では、クラス偏差の標準偏差を示した。
}
\end{flushleft}
\end{table}

\begin{figure}
\centering
\includegraphics[width=0.45\textwidth]{./fig/kasane_zmath_level.pdf}
\includegraphics[width=0.45\textwidth]{./fig/kasane_zgood_class.pdf}
\label{fig:kasane}
\caption{推定された教員付加価値の分布}
\footnotesize{
注)本図では、推定された教員付加価値の分布および子どもの教育成果の分布(ヒストグラム)を示している。青色で塗られた分布は子どもの教育成果の分布を表す。オレンジ色で塗られた分布は子どもの教育成果の分布を表す。
}
\end{figure}
% todo:6列目いらないかな。。。

% -> 教育成果の種類によって教員の影響の大きさは異なる。学力(算数・国語)を対象にした教員付加価値の標準偏差は、算数で0.013であり国語では0.003であった。これは、例えばある子どもにとって担当する教員の教員付加価値が1 標準偏差高くなったとして、子どもの算数の学力に与える影響は0.013標準偏差程度でしかない事を示している\footnote{ここで観察されるのは教員全体の子どもの学力への貢献ではなく、教員一人一人の独自の子どもの学力への貢献であるこことには注意が必要である。}。この傾向は学習方略を対象にした教員付加価値においても同様であり、その標準偏差は0.008と小さい。この傾向を日本の教員の採用や研修プロセスに帰すことができるかどうかは本稿では議論できないものの、授業の中で指導内容に教員間で差がないことが示唆される。
認知能力に与える個々の教員の影響は小さく、学習方略に対する個々の教員の影響も同様に小さい\footnote{todo。これは教員の影響が小さいこと自体は意味しない}。すなわち、(todo)列において認知能力(算数・国語)を対象にした教員付加価値の標準偏差はそれぞれ0.013・0.003であり、学習方略を対象にした教員付加価値の標準偏差は0.008である。これは元々の教育成果の標準偏差が1であることを踏まえると、子どもの認知能力や学習方略に対して個々の教員の違いの大きさは全体の変動に対して高々1\%程度しかないことを意味する。
% todo:他の研究との比較を入れたいが、、、



% -> 一方で非認知能力や学級の雰囲気に対する個別の教員の影響は比較的大きい。すなわちセルフコントロールや自己効力感・学級の雰囲気を対象にした時の教員付加価値の標準偏差の値は0.05から0.08の間である。これは生徒の教育成果の変動のうち、個別の教員の影響は5\%から8\%の間であること示す。上述の学力を対象にした時の教員付加価値の標準偏差と比較すれば、この値は大きい。
一方で、非認知能力や学級の雰囲気に対する個々の教員の影響は学力に比べ大きい。すなわち、(todo)列において非認知能力(セルフコントロール・自己効力感)を対象にした教員付加価値の標準偏差はそれぞれ0.051・0.069であり、学級の雰囲気を対象にした教員付加価値の標準偏差は0.078である。これは元々の教育成果の標準偏差が1であることを踏まえると、子どもの認知能力や学習方略に対して個々の教員の違いの大きさは全体の変動に対して5\%から8\%であることを意味する。認知能力・学習方略の変動の大きさが1\%程度であったことを踏まえると、この値は大きい。


% -> これらの結果からは教員が子どもの教育成果に対して果たしている役割をある程度素描することができる。すなわち、個別の教員は学力ではなく非認知能力や学級の雰囲気に対して重要な役割を担っている可能性がある。日本の小学校において、通塾などの外部リソースによって埋め合わせ可能な学力や学習習慣に比べれば、一日の大部分を一緒に過ごす大人として教員が子どもたちに与える人格的な影響は大きいのかもしれない。
以上の結果は、小学校において個々の教員は学力ではなく非認知能力に対してより重要な役割を担っている可能性を示唆する。例えば、教員付加価値の差が認知能力では小さい一方で非認知能力では大きいため、教員配置の変化の影響も異なりうる。子どもを担当する教員がある時変わった時に、その変化による学力の変化は大きくない一方で非認知能力の変化は比較的大きい可能性がある。この様な結果は、日本の学校教育の均質性を反映しているのかもしれない。本の教員が授業などで独自性を発揮できる余地が少なく公立小学校では学習指導要領や授業におけるガイドラインを通じて授業で行う内容が教員に依らず非常に近しいことを反映して、教員間の差が小さかった可能性がある。todo

% 教員付加価値が1標準偏差分良い教員のクラスになったところで、子どもたちの学力に与える影響は0.003標準偏差程度でしかない。
% todo:教員固定効果の場合は値が大きいことをかく。
% todo:非認知能力も影響が大きいのだから、、、
% todo:注意点としては教員ごとの違いの大きさを示しているだけだよよいうことを書く。る\footnote{ここで観察されるのは教員全体の子どもの学力への貢献ではなく、教員一人一人の独自の子どもの学力への貢献であるこことには注意が必要である。}

\subsection{教員付加価値間の相関}

% -> 次に、教員付加価値の項目間の相関について議論する。日本の初等教育においてはクラス担任制を用いている学校がほとんどである。すなわちクラスの担任になった一人の教師がほとんどの科目について勉強を教えることが多い。このような時に重要になりうるのは、ある科目において優れた教員は他の科目においても優れた教員かどうかという問題である。例えば学力を向上させる教員は、非認知能力をも向上させることができるのだろうか。学力を上昇させることに指導の重点が置かれすぎてしまうと、学力が期待通りに上がらなかった子供の自尊心が低下する可能性がある。学力を向上させるつもりで教員を雇用した結果、非認知能力に対してはかえって負の効果をうむ可能性がある。そのような項目間の教員付加価値の相関を確認した研究として\cite{goldhaber2013good}では米国ノースカロライナ州のデータをもちいて、小学校レベルでの教員の付加価値の科目間相関を検証した。その論文ではリーディング科目と数学科目の付加価値相関は0.7程度であることを報告している。同様に本節では、教員付加価値間の相関を確認する。
% -> 表\ref{table:corr}には学力・非認知能力及び学級の雰囲気について教員の付加価値間の相関を示した。そのうち特に重要な点として3点指摘することができる。

次にある教員における複数の教員付加価値間の関係性について考えるため、表\ref{table:corr}には教員付加価値間の相関を示した。この相関を確認することで教員の能力がどの様なものであるかについて推察する事が可能になり、さらに教員の配置に関する知見を得る事ができる。教員付加価値が高い事や低い事はtodo.
% todo(行と列を示す、)


\begin{table}[htbp]
\begin{adjustbox}{max width=1\textwidth}
\input{table/corr_va_outcome.tex}
\end{adjustbox}
\caption{教員付加価値の項目間相関}
\label{table:corr}
\begin{flushleft}
\footnotesize{
注)本表では教員付加価値の項目間の相関を示している。各々のセルは相関係数を表す。各々の行及び列は教員付加価値を求める対象とした教育成果を表す。「***」・「**」・「*」はそれぞれ1\%・5\%・ 10\%の水準で統計的に有意であることを表す。
}
\end{flushleft}
\end{table}

% -> 第一に、国語と算数の教員付加価値の間の相関は統計的に有意であり、相関関係数も0.481と正である。すなわち、国語を教える際に高いパフォーマンスを発揮している教員は算数でも高いパフォーマンスを発揮している。高いパフォーマンスを発揮している教員は、ある特定の科目を教える事が得意というよりむしろ、どのような科目であっても知識を伝達することを得意としているのかもしれない。
第一に子どもの国語の学力を向上させた教員は、算数の学力をも向上させた傾向がある。すなわち、国語と算数の教員付加価値の間の相関(todo列、todo行)は統計的に有意であり、相関関係数も0.481と正である。すなわち、国語を教える際に高いパフォーマンスを発揮している教員は算数でも高いパフォーマンスを発揮している。高いパフォーマンスを発揮している教員は、ある特定の科目を教える事が得意というよりむしろ、どのような科目であっても知識を伝達することを得意としているのかもしれない。


% -> 第二に、学習方略と非認知能力(セルフコントロール・自己効力感)・学級の雰囲気の教員付加価値はその間に統計的に有意な相関があり\footnote{セルフコントロールと自己効力感の教員付加価値の間には有意な相関はないが、これはサンプルが非常に少ないからだと考えられる}、相関係数も正である。子どもたちの学習習慣を向上させる事ができる教員は担当する学級の雰囲気も良く、最終的に非認知能力も向上させることができる。
第二に子どもの非認知能力を向上させた教員は、学習方略や学級の雰囲気をも向上させた傾向がある。学習方略と非認知能力(セルフコントロール・自己効力感)・学級の雰囲気の教員付加価値はその間に統計的に有意な相関があり\footnote{セルフコントロールと自己効力感の教員付加価値の間には有意な相関はないが、これはサンプルが非常に少ないからだと考えられる}、相関係数も正である。子どもたちの学習習慣を向上させる事ができる教員は担当する学級の雰囲気も良く、最終的に非認知能力も向上させることができる。todo詳しく

% -> 第三に、学力(国語・算数)の教員付加価値と学習方略・非認知能力の教員付加価値の間の相関は小さく統計的な有意性もない。例えば、算数を対象とした教員付加価値とセルフコントロールを対象にした教員付加価値の間の相関係数は0.093と小さく統計的に有意ではない。すなわち、学力を伸ばすという意味で「良い」教員と学習習慣や非認知能力を伸ばすという意味で「良い」教員は必ずしも同一ではない。子どもの学力と非認知能力・学習環境の間には強い相関があることを踏まえれば、この関係性は教員付加価値特有の性質であると考えられる。子どもの学力を伸ばすために教員に求められるスキル・能力と、非認知能力を伸ばすために教員に求められるスキル・能力は非常に異なるものなのかもしれない。
% todo 出展追加->子どもの学力と非認知能力・学習環境の間には強い相関があることを踏まえれば
第三に子どもの学力を向上させたからといって、非認知能力をも向上させたわけではないことが分かった。
第三に、学力(国語・算数)の教員付加価値と学習方略・非認知能力の教員付加価値の間の相関は小さく統計的な有意性もない。例えば、算数を対象とした教員付加価値とセルフコントロールを対象にした教員付加価値の間の相関係数は0.093と小さく統計的に有意ではない。すなわち、学力を伸ばすという意味で「良い」教員と学習習慣や非認知能力を伸ばすという意味で「良い」教員は必ずしも同一ではない。子どもの学力と非認知能力・学習環境の間には強い相関があることを踏まえれば、この関係性は教員付加価値特有の性質であると考えられる。子どもの学力を伸ばすために教員に求められるスキル・能力と、非認知能力を伸ばすために教員に求められるスキル・能力は非常に異なるものなのかもしれない。





% -> ただし両者の関係性はトレードオフというわけではなく、また前節で議論した様に学力における個々の教員の貢献というのは非常に小さいと考えられる。そのため、教員の雇用や配置といった視点で考えた時には、非認知能力の向上を得意とする教員を用いることがより効果的でありうるかもしれない。
これらの結果をまとめると、教員の能力は多次元的である事がわかった。一言に「良い」教員がいるわけではなく、学力については良いけど、、、みたいな教員もいるよね。todo
学力と非認知能力の教員付加価値の間にトレードオフがあるわけではないので、非認知能力を向上させることができる教員を任用することは子どもの人的資本の向上という意味で有効でありうる。




\subsection{教員の属性と教員付加価値}
% -> ここまで推定された教員付加価値そのものを分析してきたが、それではどのような教員が高い教員付加価値を持つのだろうか。多くの研究が観察可能な教員の特徴が必ずしもお子どもの教育成果とは結びつかないことを報告している\cite{hanushek2006teacher}。しかしその多くは、教員の観察可能な特徴と学力の間の関係性を考察するものがほとんであり、さらに日本の教員を対象にした研究を考えれば一部の例外を除けば\citep{二木美苗2017子ども}非常に少ない。
% -> 本節では教員の観察可能な特徴と教員付加価値の間にある関係性について分析を行う。分析にあたっては下記の様なOLSによる分析を行う。

 \begin{equation*}
  \hat{\mu}_{j} = \beta x_{j, 2018}+ \epsilon_{j}
   \label{eq_persist}
 \end{equation*}

% $ \hat{\mu}_{j} $は上記の手続きで推定された教員付加価値である。$x_{j, 2018}$は観察可能な教員の特徴(2018年度)であり、具体的には教員の年齢や教員歴、性別、出身大学や所属する学校を表す。特に年齢などの年によって変動する値については、すべて2018年度時点での値を用いた。推定された$\hat{\beta}$が本節で関心を寄せる値である。
本節では教員付加価値が高い教員の具体的な特徴について考えるために、教員の観察可能な特徴と教員付加価値の間にある関係性について考える。多くの研究が観察可能な教員の特徴が必ずしも子どもの教育成果とは結びつかないことを報告しており\cite{hanushek2006teacher}、研究間で一貫して効果的だとされる具体的な要因は見つかっていない。しかし、それらの研究はほとんど学力を対象にしたものであり、ここまで考えてきた非認知能力や学級の雰囲気などの学力以外の教育成果と教員の観察可能な特徴についての研究はほとんど存在しない。さらに、そもそも日本の教員を対象にした場合、\ref{litelature}でも述べたとおりほとんど研究が存在しないため海外での研究での結論を当てはめることが可能なのかどうかは分からない。そのため、教員質問紙で尋ねている教員の特徴と推定した教員付加価値の間の相関について分析していく。


分析にあたっては下記の様な定式化で分析を行う。観察可能な特徴として、具体的には教員の年齢、教員歴、性別、出身大学の難易度を考える。
\begin{equation*}
  \hat{\mu}_{j} = \beta x_{j, 2018}+ \epsilon_{j}
   \label{eq_persist}
\end{equation*}
$ \hat{\mu}_{j} $は上記の手続きで推定された教員付加価値である。$x_{j, 2018}$は観察可能な教員の特徴(2018年度)であり、具体的には教員の年齢や教員歴、性別、出身大学や所属する学校を表す。特に年齢などの年によって変動する値については、すべて2018年度時点での値を用いた。推定された$\hat{\beta}$が本節で関心を寄せる値である。

\subsubsection*{年齢、教員歴}
どの様な教員が良い教員か考えた時に頻繁に挙げられる点として、年齢や教員歴といった仕事の経験の多寡をあげることができる。教員という仕事は高度な専門職でありその業務経験が長ければ長いほど教員としてのスキルは向上する可能性がある。\cite{hanushek2006teacher}では教員の年齢が子どもの教育成果に与える影響を分析した研究をサーベイし、研究によってその推定値は様々であり結論をいまだに出せるような状況ではないとしている。一方で、ほとんどの研究が学力を対象にした効果の分析を行なっているが、年齢や教員歴の効果は何の教員付加価値を用いて評価するかによって異なる可能性がある。

推定結果を表\ref{table:age}に示した。非線形性を考慮するために年齢の二乗項を推定式に加えて推定を行った。
% 細かい条件も書いていく
% 表\ref{table:age}の(4)列みたいな書き方
% todo また視覚的に確認するために、一部の教員付加価値については教員の年齢と教員付加価値の間の関係性をプロットしたものを図\ref{todo}に示した。図\ref{todo}にはNadaraya Watson 推定量を同時に示した。

\begin{table}[htbp]
\begin{adjustbox}{max width=1\textwidth}
\input{table/va_tinfo_age.tex}
\end{adjustbox}
\caption{年齢及び教員歴が教員付加価値に与える影響}
\label{table:age}
\begin{flushleft}
\footnotesize{
注)本表では年齢及び教員歴が教員付加価値に与える影響をを示している。各々のセルでは推定値及び標準誤差・有意水準を示しており、「推定値 有意水準 (標準誤差)」という表記法を用いている。「***」・「**」・「*」はそれぞれ1\%・5\%・ 10\%の水準で統計的に有意であることを表す。
}
\end{flushleft}
\end{table}

表\ref{table:age}からわかる通り、教員の年齢はどの教員付加価値との間にも統計的に有意な関係性はなかった。ただし推定値自体は正であり、かつサンプルサイズが小ない事を踏まえれば、教員の年齢と教員付加価値の間に全く関係性がないとするのは早急かもしれない。

教員歴についても概ね同様の傾向であった。ただし、自己効力感および学級の雰囲気についての教員付加価値との間に、教員歴はそれぞれ5\%および10\%の水準で有意な正の相関があった。一方で、教員歴の係る推定値は統計的な有意性はなくとも、すべて正であった。

総じて推定値自身は概ね正の値をとっているものの概ね統計的に有意な水準ではない。そのため、教員の年齢や教員歴と教員付加価値との間に強い関係性を見つけることはできなかったと言えよう。
% その事は図{todo}などで視覚的に確認してもまた同様である。

\subsubsection*{性別}

教員の性別も教員の能力という観点では頻繁に焦点となる。小学校での女性教員比率は日本全体で60\%であり(2010年)、教員の割合が大きい。日本の教員という職業は性別による賃金差が比較的小さい\citep{senoo2003}ため、相対的に労働市場で能力の高い女性労働者が教員になっている可能性がある。
% (todo出展)

表\ref{table:sex}には教員の性別と教員付加価値の間にある関係性を示している。結果として、教員が女性であるか否かと教員付加価値との間には概ね正の関係は存在するものの、その推定値の分散は大きい。例外的に、算数についての教員付加価値については、推定値は0.004と10\%の水準で有意であった。この推定値の水準は教員付加価値の標準偏差が0.013(表\ref{table:variance})である事を踏まえれば非常に高いものの、依然として慎重な議論が必要である。

\begin{table}[htbp]
\begin{adjustbox}{max width=1\textwidth}
\input{table/va_info_sex.tex}
\end{adjustbox}
\caption{性別及び出身大学偏差値が教員付加価値に与える影響}
\label{table:sex}
\begin{flushleft}
\footnotesize{
注)本表では性別及び出身大学偏差値が教員付加価値に与える影響をを示している。各々のセルでは推定値及び標準誤差・有意水準を示しており、「推定値 有意水準 (標準誤差)」という表記法を用いている。「***」・「**」・「*」はそれぞれ1\%・5\%・ 10\%の水準で統計的に有意であることを表す。
}
\end{flushleft}
\end{table}

% Eide et al. (2004) suggest a possible explanation of the potential performance difference between male and female teachers First, they show the portion of female teachers are much higher in elementary school, and secondary school than male ones. They also show that teachers are adversely selected, i.e. not the most talented persons choose to become teachers, and many of them leave their job seeking higher wages. This argument is widely accepted in the literature. Bacorod (2003), Dolton, (2006), Dolton-Marcenaro-Gutierrez, (2011), Varga (2017) also provides indirect evidence for adverse selection into the teacher profession

\subsubsection*{出身大学の偏差値}

高い学力・認知能力を持つ教員は高いパフォーマンスを発揮するのではないかと言う仮説は頻繁に検証されてきた。例えば\cite{二木美苗2017子ども}など教員の採用時の労働市場一般における指数が子どもの教育成果に与える影響を調べる研究は、暗に能力の高い労働者ならば教員としても能力が高い可能性があるという仮説を検証していると言えよう。\cite{hanushek2006teacher}ではそのサーベイを行い、最終的には研究によってその推定値は様々であり結論をいまだに出せるような状況ではないとしている。

本稿で用いるデータでは、2016年度にT市に所属した教員に対する質問紙調査の中で卒業大学及び卒業学科について尋ねている。それらの卒業大学の情報を2018年度に河合塾が算出した大学受験偏差値に紐付けることで、教師の学力を卒業大学の偏差値という形で取得した。無論入試難易度偏差値は時間によって変わるため、このひも付けは必ずしも適切ではないことに十分注意する必要がある。

表\ref{table:sex}には教員の卒業大学偏差値と教員付加価値の間にある関係性を示している。結果として、卒業大学偏差値が教員付加価値に与える影響は概ね正である一方でその推定値の分散は大きく統計的に有意ではなかった。

\section{結論\label{conclusion}}
本稿では子どもの教育成果に対する教員の効果について考えた。教員効果を教育成果に対する付加価値として計測した上で、得られた推定値の性質を記述的に調べた。

まず教員が子どもの教育成果に与えている影響の大きさは、教育成果の種類に応じて異なることがわかった。学力や学習習慣に対する教員の影響は小さかった一方で、非認知能力や学級の雰囲気に対する教員の影響は比較的大きかった。そのため、「教員は子どもに大きな影響を与える/与えない」といった言説について評価するときには、子どもの人的資本のどの部分を対象にしているかを考える必要がある。子どもの学力の向上の期待して能力の高い教員を雇用することは必ずしも効果的な施策ではないかもしれないが、非認知能力や学級への雰囲気への効果を期待して能力の高い教員を雇用することは高い効果を持ちうる。次に教員付加価値間の相関を調べた結果、学力を伸ばすことが得意な教員と学習方略や非認知能力を伸ばすことが得意な教員は異なるのではないかという示唆を得た。両者の間で教員として必要な能力が異なるのかもしれない。そして、この結果を踏まえた時に、教員付加価値を用いて教員の評価をする事については慎重に検討するべきであると考えられる。すなわち、教員を評価するとき一概に「良い」か「悪い」の2つに分けるのが難しい。学力において良いパフォーマンスを発揮した先生であっても、非認知能力などでは必ずしもその様ではない。

その後、高い教員付加価値を持った教員の具体的な特徴を調べた。しかし、年齢や教員歴・性別や学力について調べたが、頑健な関係性を得ることはできなかった。\cite{hanushek2006teacher}などで指摘されている通り、教員の能力と観察可能な教員の特徴と結びつけることは未だに困難があるのかもしれない。

本稿の最大の限界はサンプルサイズの問題である。特に教員のユニークな数は少なく、解析の結果がどれほど頑健であるかという点に課題が残る。本稿では市町村に在籍する教員全ての情報を使っているため、もとより教員のパフォーマンス測定の文脈で教員固定効果を考える時にはより大きい行政区分単位で考えなければいけない可能性がある。

また本項で得られた教員の付加価値は、因果効果とも異なることに注意が必要である。\cite{chetty2014measuring}らの議論の様に得られた教員付加価値が因果効果を持ちうることを示した研究は幾つか存在し、本稿はそれらの議論を踏まえて推定された教員付加価値をそのまま用いている。しかし、日本においても教員付加価値が因果的な効果を持ちうるかどうかは必ずしも自明ではなく、検証を必要とする。

\bibliographystyle{jecon}
\bibliography{kinsler.bib}

\appendix

\end{document}

